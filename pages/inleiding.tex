	\chapter{Inleiding}
	[comments]
	Leg in de inleiding van breed naar smal het probleem in zijn context uit.\\
	Je eindigt de inleiding precies daar waar je het probleem globaal beschreven hebt.\\
	Dan in het hoofdstuk specificatie geef je alle onderdelen weer die door fabrikanten aangeleverd worden, voor zover van toepassing. (Let op! Dit is een keuze, vaak is er een wisselwerkng tussen je specificatie en je requirements en daarmee de volgorde van de hoofdstukken.)\\
	Nu komt de probleem stelling met eventuele onderzoeksvragen aan de orde met daarbij de requirements, etc. Een indeling kun je ook op BB vinden. Die is echter niet zaligmakend,\\ aangezien het kan zijn dat het probleem op zich leidt tot onderzoeksvragen die aanleiding geven tot een hardware keuze.\\
	
	Uiteindelijk moet in je verslag een hoofdstuk het logische vervolg zijn op het vorige hoofdstuk.\\
	

	Geef ter illustratie in dit hoofdstuk ook schematische weergaves van de hardware (het NI-FPGA bord en het RM88-N bord).\\
	Hiernaar kun je dan in je tekst verwijzen.
	
	
	
	