\chapter{Implementatie}

Voordat er begonnen kan worden wordt er eerst een korte uitleg gegeven over de structuur, syntax en code van de hardware beschrijving taal VHDL.

\section{VHDL uitgelegd}

VHDL begint met een declaratie van alle gebruikte libraries, dit zijn bibliotheken waarin staat hoe de VHDL code met het XILINX bord om moet gaan, bijvoorbeeld waar de aansluitingen zitten.

Na het declareren van de libraries moeten alle IO poorten van het systeem gedeclareerd worden.

Na het declareren van de libraries en de IO poorten kan er nog gebruikt gemaakt worden van interne variabelen, signals genoemt. Waar de IO poorten alleen binair 0 of 1 kunnen zijn kunnen signals meer informatie bevatten, het is mogelijk om integers, bools en dergelijke te gebruiken.

Nadat alles gedeclareerd is kan er begonnen worden met de daadwerkelijke functies.

\clearpage
\section{Declaratie VHDL}

De volgende libraries moeten gedeclareerd worden om de VHDL te kunnen communiceren met het XILINX bord.

\begin{itemize}
	\item $ieee;$
	\item $ieee.std_logic_1164.all;$
	\item $IEEE.std_logic_unsigned.all;$
	\item $ieee.numeric_std.all;$
\end{itemize}
\section{Initialisatie van het schuifregister}

Hieronder wordt een lijst gegeven van alle $\boldsymbol{uitgangen}$ die gebruikt gaan worden:
\begin{itemize}
	\item LED0 t/m  LED7.\\
	Deze LEDs worden gebruikt om de toestand van de uit Data-Out ingelezen bits te weer te geven, ofwel 1(Led aan) ofwel 0(led uit).
	\item GPIO16\\
	Dit is de IO poort waarop Load aangesloten wordt.
	\item GPIO17\\
	Dit is de IO poort waarop Reset aangesloten wordt.
	\item GPIO14\\
	Dit is de IO poort waarop Clock aangesloten wordt.

\end{itemize}

Hieronder wordt een lijst gegeven van alle $\boldsymbol{Ingangen}$ die gebruikt gaan worden:
\begin{itemize}
		\item GPIO12\\
		Dit is de IO poort waarop Data-Out aangesloten wordt.
		\item SW0\\
		Dit is de IO poort die aangeeft of de upper of lower byte gelezen wordt.
		\item Onboardclock\\
		Dit is de eerdergenoemde interne clock, deze is geklokt op $\SI{50}{\mega\hertz} $. 
\end{itemize}


\section{Uitlezen van het Schuifregister}

